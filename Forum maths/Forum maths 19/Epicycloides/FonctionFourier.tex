\documentclass[10pt,a4paper]{article}
\usepackage[utf8]{inputenc}
\usepackage{amsmath}
\usepackage{amsfonts}
\usepackage{amssymb}
\usepackage[left=2cm,right=3cm,top=2cm,bottom=2cm]{geometry}

\begin{document}
\begin{center}
\begin{LARGE}Fonction issue de la transformée de Fourier discrète appliquée à la suite de points dessinée.
\end{LARGE}
\end{center}

On a rangé les coefficients du plus grand au plus petit (en module).
\vspace{2em}

$f(t)=419.18+244.79i+(-36.8-122i)\textbf{e}^{-it}+(14.1+37.1i)\textbf{e}^{-2it}+(9.2-20i)\textbf{e}^{it}+(-7.35+15.8i)\textbf{e}^{2it}+(4.6+2.67i)\textbf{e}^{-3it}+(-4.49-1.92i)\textbf{e}^{-4it}+(4.02-0.726i)\textbf{e}^{6it}+(-3.77+1.04i)\textbf{e}^{7it}+(-3.76+0.377i)\textbf{e}^{-9it}+(3.21+0.107i)\textbf{e}^{-8it}$

\end{document}